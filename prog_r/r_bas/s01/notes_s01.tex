% Preámbulo
\documentclass{article}
    % UTF-8 encoding
    \usepackage[utf8]{inputenc}
    % Idioma
    \usepackage[spanish]{babel}
    % Geometría
    \usepackage[a4paper]{geometry}
    % Título | autor | fecha | cambios
    \title{Notas de clase de la sesión 1}
    \author{
        Jhon R. Ordoñez \\
        \textcolor{blue}{rolyordonezleon@gmail.com}
    }
    \date{\today}
    \usepackage{titling}
        \setlength{\droptitle}{-1.5cm}
        \setlength{\parindent}{0cm} % 1.27cm APA7
    % Encabezado | pie de página
    \usepackage{fancyhdr}
        \pagestyle{fancy}
        \fancyhf{} % Limpiar encabezado y pie de página predefinido
        \lhead{2023}
        \chead{Programación en R} 
        \rhead{Jhon R. Ordoñez}
        \cfoot{\thepage} 
    % Textos automático
    \usepackage{lipsum}
    % Colores
    \usepackage{xcolor}
    \usepackage{tcolorbox}
    % Personalización para ingresar el código de R
    \usepackage{listings}
    \lstset{
    	language=R,
    	basicstyle=\ttfamily,
    	numbers=left,
    	numberstyle=\tiny,
    	stepnumber=1,
    	numbersep=5pt,
    	backgroundcolor=\color{white},
    	showspaces=false,
    	showstringspaces=false,
    	showtabs=false,
    	tabsize=2,
    	captionpos=b,
    	breaklines=true,
    	breakatwhitespace=false,
    	title=\lstname,
    	keywordstyle=\color{blue},
    	commentstyle=\color{green},
    	stringstyle=\color{red},
    	morekeywords={install.packages, library, data.frame, plot},
    }
	% Ingresar código sin generar un error
	\usepackage{verbatim}
	
% Cuerpo
\begin{document}
    % Hacer el título
        \maketitle
    % Estilo de la página
        \thispagestyle{fancy}
    % Sección 1
    \section{Vectores y matrices}
       % Subsección 1.1
       \subsection{Vectores}
       Creamos los vectores \textbf{peso} y \textbf{edad} en la misma dimensión.
\begin{lstlisting}
# Creamos los vectores peso y edad
peso <- c(4.4, 5.4, 6.4, 3.2, 7.5, 3, 6.1, 3.1, 6.1, 7, 3.4)
edad <- c(1, 2, 3, 4, 5, 6, 7,8, 9, 1)
\end{lstlisting}
    {\bf \it Nota.- } En $R$ los índices (numeraciones) comienzan en el $1$, no en el $0$ como ocurre en muchos lenguajes de programación. \\
    
    {\bf \it Extra:} El operador “:” genera secuencias.
\begin{lstlisting}
# Ejemplo
c(2:6)
\end{lstlisting}    
    	Creamos los vectores $x$, $y$  y $z$.
\begin{lstlisting}
# Vectores
x <- c(1, 2, 3, 4, 5, 6, 7, 8)
y <- c("juan", "pepe", "inaky", "amparito", "mariano",
	   "juancar", "fulano", "elefante")
z <- c(TRUE, TRUE, FALSE, TRUE, FALSE, FALSE, 
       TRUE, TRUE)
\end{lstlisting}    	
   
	% Subsección 1.2
	\subsection{Matrices}     
		Una matriz es un vector con un atributo adicional (\verb|dim|), que a su vez es un vector numérico
		de longitud 2 que define el número de filas y columnas. Se crean con la función \verb|matrix()|.

\begin{lstlisting}
matrix(data = NA, nrow = 2, ncol = 2,
       byrow = F, dimnames = NULL) 
matrix(data = 1:4, nrow = 2, ncol = 2,
       byrow = F, dimnames = NULL)
matrix(data = 5, nrow = 2, ncol = 2, 
       byrow = F, dimnames = NULL)
\end{lstlisting}

	% Subsección 1.3
	\subsection{Data frames} 	
		Un dataframe (a veces se traduce como ‘marco de datos’) es una generalización
		de las matrices donde cada columna puede contener tipos de datos distintos al
		resto de columnas, manteniendo la misma longitud. Es lo que más se parece a una
		tabla de datos de $SPSS$ o $SAS$, o de cualquier paquete estadístico estándar.
		Se crean con la función \verb|data.frame()|.
		
\begin{lstlisting}
# Ejemplo. data.frame() 
	diabetes <- c("Tipo1", "Tipo1", "Tipo2", "Tipo2",
	              "Tipo1", "Tipo1", "Tipo2", "Tipo1")
	estado <- c("bueno", "malo", "bueno", "bueno",
	            "bueno", "malo", "bueno", "malo")
\end{lstlisting}

		{\bf \it Extra:} Crear un tabla cruzada.
		
\begin{lstlisting}
# Creamos un tabla cruzada.
	table(df$diabetes, df$estado)
\end{lstlisting}		
		
\end{document}

